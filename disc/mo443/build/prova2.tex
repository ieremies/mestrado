% Created 2023-06-19 seg 17:42
% Intended LaTeX compiler: pdflatex
\documentclass[twocolumn, 9pt]{article}
\usepackage[utf8]{inputenc}
\usepackage[T1]{fontenc}
\usepackage{graphicx}
\usepackage{longtable}
\usepackage{wrapfig}
\usepackage{rotating}
\usepackage[normalem]{ulem}
\usepackage{amsmath}
\usepackage{amssymb}
\usepackage{capt-of}
\usepackage{hyperref}
\usepackage{todonotes}
\usepackage[brazil, american]{babel}
\usepackage{amsthm}
\usepackage[a4paper, total={7in, 9in}]{geometry}
\author{Ieremies Vieira da Fonseca Romero}
\date{}
\title{Prova 2}
\hypersetup{
 pdfauthor={Ieremies Vieira da Fonseca Romero},
 pdftitle={Prova 2},
 pdfkeywords={},
 pdfsubject={},
 pdfcreator={Emacs 28.2 (Org mode 9.6.1)}, 
 pdflang={English}}
\usepackage[date=year]{biblatex}
\addbibresource{~/arq/bib.bib}
\begin{document}

\maketitle

\section*{Slides}
\label{sec:orgc5de01d}
\subsection*{Cores}
\label{sec:org14fa4aa}
A formação de cores se dá por meio de dois processos:
\begin{description}
\item[{aditivo}] energias dos fótons são combinados
\item[{subtrativo}] quando a luz passa por um meio que filtra certas frequências
\end{description}

As características das cores são \emph{brilho}, \emph{matiz} e a \emph{saturação}.
\begin{description}
\item[{brilho}] ou liminância representa a noção de intensidade da radiação (claro ou escuro)
\item[{matiz}] associada ao comprimento de onda (azul ou vermelho)
\item[{saturação}] pureza do matiz, grau de mistura do original com a luz branca (cores puras são completamente saturadas)
\end{description}

Matiz + saturação são chamados de \emph{crominância}.

\subsubsection*{Modelos de cores}
\label{sec:orgc6e6d64}
Sistemas de representação tri-dimensional das cores, usando as caracterîsticas acima.
O espaço de cores possíveis num modelo é chamado de gamute.

Em modelos aditivos, a cor branca é a soma de todas as cores.
Em modelos subtrativos, é a ausência (ausência da filtragem).

O modelo CMY, subtrativo, é usado em dispositivos de pigmentação e, para evitar o consumo excessivo de tinta, muitas vezes adiciona o preto, formando o CMYK.
O modelo HSV (hue, saturation, value) forma uma pirâmide hexagonal, na qual a base é composta das 3 cores primárias e 3 secundárias e a altura é a luminância (sendo a base a mais clara).
O modelo HSL forma dois cones de altura 1 total, onde uma das pontas é o preto e outra o branco.
O meio, largo ficam as cores.

\subsection*{Textura}
\label{sec:org48121c0}
Finas :: interações aleatórias e grandes variações
Ásperas :: interações melhor definidas e regiões homogêneas

Extração procura retirar um quantidade de dados representativa e simplificada, enquanto a seleção visa reduzir o número de medidas

\subsubsection*{Matriz de coocorência}
\label{sec:org9ab4d2b}
Matriz definida a partir de uma relação entre pixels (por exemplo, visinho ao lado), onde a posição i,j da matriz é a quantidade de pares na relação que possuem uma transição do nível de cinza i para j.
Utilizamos esta matriz na forma normalizada (dividindo cada elemento pelo número total de transições).

Para generalizar, podemos definir a relação entre pixels a partir da distância e ângulo (zero graus é direita) entre eles.
Qualquer alteração nessas medidas altera de forma significativa na matriz, a qual não tem mais informações espaciais sobre os pixels originais.

\begin{description}
\item[{Segundo momento angular}] (energia) expressa a uniformidade de uma textura \(f_{sma} = \sum \sum p_{i,j}^2\).
Em texturas ásperas, poucos elementos da coocorencia normalizada apresentam valores diferentes de zero e, quando ocorrem, são próximos de um e o segundo momento angular apresenta valor próximo a 1.
\item[{Entropia}] quando uma imagem não é uniforme, as entradas \(p_{i,j}\) apresentam valores próximos a zero e \(f_{ent} = - \sum \sum p_{i,j} log(p_{i,j})\) em valorea altos não normalizados.
\item[{Contraste}] diferença entre tons de cinza, baixo ocorre quando há pequena diferença entre os níveis  em uma região contínua \(f_{con} = \sum \sum (i-j)^2 p_{i,j}\).
\item[{Heterogeneidade}] 

\item[{Correlação}] 

\item[{Homogeneidade}] 
\end{description}

\subsubsection*{Matrizes de comprimento de corridas de cinza}
\label{sec:orgefbd0f9}
Matriz dada por \(P(i,j|\theta)\) contém a quantidade de corridas do mesmo nível de cinza \(i\) e comprimento \(j\) na direção \(\theta\).
Uma corrida de tamanho 4 e 21 de tamanho -> fina.
Várias corridas de tamanho grande -> expessa

\subsubsection*{Função de autocorrelação}
\label{sec:org6cd1d50}
Finas -> primitivas pequenas -> frequências especiais altas
A função de auto correlação descreve as interações epsaciais entre as primitivas
\subsubsection*{Unidade de textura}
\label{sec:orgfdacddc}
Definida a partir da relação do pixel central com seus 8 vizinhos. Para cada um deles, olhamos se ele é menor, igual ou maior que o valor do pixel central. Associamos 0, 1, 2 respecitivamente e definimos a unidade como o polinômio a\textsubscript{1} 3\textsuperscript{0} +  a\textsubscript{2} 3\textsuperscript{1}\ldots{} um número na base 3, Perceba que a ordem importa!

LBP Padrões locais binários usa a mesma ideia só que zeros e uns
\subsection*{Registro}
\label{sec:org0e52fa1}
Trans formações geométrias: transformação espacial (reorganização dos pixels no plano) + interpolação de intensidade.

Mapeamento direto: da original para a transformada, podendo mais de um pixel cair no mesmo lugar
Mapeamento indireto: usa-se a inversa, aplicando-a transformada, mas que faz pixels da resultante serem mapeados ao mesmo da original

Coordenadas homogêneas: para permitir que as transformações espaciais possam ser realizadas por meio de multiplicação de matrizes e que possa haver combinação delas (x,y,z) -> (Wx, Wy, Wz, W).

\begin{center}
\includegraphics[width=.9\linewidth]{/home/ieremies/org/.attach/f9/12474e-2e4a-43f1-959a-b8ac36080d6e/_20230619_165656screenshot.png}
\end{center}
\subsubsection*{Transformações afim}
\label{sec:org71574a5}
generalizam transf. de rotação, translação, escala, reflexão e cisalhamento.
Preservam o paralelismo e a proporção entre volumes, áreas e comprimentos.
\begin{center}
\includegraphics[width=.9\linewidth]{/home/ieremies/org/.attach/f9/12474e-2e4a-43f1-959a-b8ac36080d6e/_20230619_165716screenshot.png}
\end{center}
\begin{itemize}
\item Mudança de escala
\label{sec:org2a22374}
\begin{center}
\includegraphics[width=.9\linewidth]{/home/ieremies/org/.attach/f9/12474e-2e4a-43f1-959a-b8ac36080d6e/_20230619_170134screenshot.png}
\end{center}
\item Translação
\label{sec:orgf02c3de}

\begin{center}
\includegraphics[width=.9\linewidth]{/home/ieremies/org/.attach/f9/12474e-2e4a-43f1-959a-b8ac36080d6e/_20230619_170202screenshot.png}
\end{center}

\item Rotação
\label{sec:orgaeda7d2}
Em 2d é dado pela matriz
cos -sen
sen cos
\end{itemize}
\subsubsection*{Projeções}
\label{sec:org1c04c8c}
Cada um dos pontos que formam uma cena no espaço tridimensional possa ser projetado no plano de imagem.
\begin{description}
\item[{Ortográfica}] pontos são projetados ao longo de linhas paralelas na imagem, projetamos em um dos plano. A matriz é a identidade com coeficiente zero no plano a ser projetado.
\item[{Perspecitva}] tamanho dos objetos reduz conforme a distância. Lembre-se de ótica do EM.
O centro da lente fica no eixo z, a uma distância \(f\) focal da origem.
\end{description}
\begin{center}
\includegraphics[width=.9\linewidth]{/home/ieremies/org/.attach/f9/12474e-2e4a-43f1-959a-b8ac36080d6e/_20230619_171103screenshot.png}
\end{center}
\begin{center}
\includegraphics[width=.9\linewidth]{/home/ieremies/org/.attach/f9/12474e-2e4a-43f1-959a-b8ac36080d6e/_20230619_171041screenshot.png}
\end{center}

\begin{center}
\includegraphics[width=.9\linewidth]{/home/ieremies/org/.attach/f9/12474e-2e4a-43f1-959a-b8ac36080d6e/_20230619_172529screenshot.png}
\end{center}
\subsubsection*{Interpolação}
\label{sec:orgbb99fc4}
\subsubsection*{Técnicas de registro}
\label{sec:orgfae01bb}
Iterativo, correlação de fase (fourrier)

\subsection*{Compressão}
\label{sec:orge460ab4}
\begin{description}
\item[{Sem perda}] imagens cujos dados são de difícil aquisição.
\item[{Com perda}] nem toda informção é recuperada, mas tá tudo be,
\end{description}

Em geral, as técnicas se baseam na redução de redundâncias.
Redundância é medido como a parte que foi jogada fora na compressão.
Uma compressão de 10:1, joga 90\% fora.
Informação basea-se na capacidade de obter significado.
\begin{description}
\item[{Redundância de codificação}] Seja \(\bar{L}\) o comprimento médio de bits para representação de um pixel, uma imagem de MxN possui codificação ótima com \(MN\bar{L}\) bits.
Um código é determinado ótimo se seu comprimento mínimo é \(\bar{L}\).
\item[{Redundância interpixel}] Pixels visinhos possuem valores próximos.
Armazenar seus valores absolutos pode gastar mais espaço que só armazenar a diferença entre eles.
\begin{description}
\item[{Codificação por comprimento de corrida}] (valor, quantidade\textsubscript{dele}\textsubscript{consecutivas}) para cada linha
\end{description}
\item[{Redundância psicovisual}] Podemos remover algumas informações que o olho humano tende a não dar atenção.
\end{description}

\subsubsection*{Teoria da informação}
\label{sec:org17d8350}
A informação obtida a partir de um evento aleatório é dado por \(I(E) = log_b \frac{1}{P(E)} = -log_b P(E)\).
Se um evento sempre/nunca ocorre, não há nenhuma informação a ser obtida.
Quanto mais improvável, maior a quantidade de informação a é necessário para comunicar o evento.

Entropia é \(E = - \sum p_i log_2 p_i\)
A eficiência da codificação pode ser definida como \(n = E/\bar{L}\)

\subsubsection*{Métodos de compressão}
\label{sec:org91cc860}
\begin{itemize}
\item Sem perdas
\label{sec:orgee78416}
\begin{itemize}
\item Huffman
\label{sec:org025d030}
Podemos usar códigos de tamano variável desde que nenhum código seja prefixo de outro de tamanho maior.
Para determinar os códigos, utilizamos a técnica de redução de fonte:
\begin{itemize}
\item ordenamos os símbolos por probabilidade
\item a cada passo, combinamos os dois com menor probabilidade, somando-as.
\end{itemize}
\begin{center}
\includegraphics[width=.9\linewidth]{/home/ieremies/org/.attach/bc/4bf780-600f-41d9-893b-752fce8e3170/_20230619_122001screenshot.png}
\end{center}
\begin{itemize}
\item Depois, retornamos, da direita à esquerda atribuindo códigos às probabilidades. Adiciona-se um bit a cada símbolo préviamente agrupado
\end{itemize}
\begin{center}
\includegraphics[width=.9\linewidth]{/home/ieremies/org/.attach/bc/4bf780-600f-41d9-893b-752fce8e3170/_20230619_122042screenshot.png}
\end{center}

As vezes, resolver esse processo pode ser computacionalmente complexo ou gerar código proibitivamentes longos.
Uma solução para isso é aplicar à apenas os \(m\) símbolos mais frequêntes enquanto o resto usa um prefixo livre e tamanho fixo.
\item Shannon-Fano
\label{sec:org1f646ad}
Divisão e conquista, cada divisão um fica com 0 e outro com 1.
\item Dicionário
\label{sec:org90b4eba}
LZ78: vou adicionando simbolos à minha palavra enquanto a palavra resultante ainda aparecer no dicionário. Quando isso deixar de ser verdade, eu uso o código da maior palavra que consegui e começo de novo.

LZW: Iniciamos com um dicionário com todas as palavras de um símbolo. A cada passo, adicionamos o próximo simbolo c a nossa palavra I. Se I + c existe no dicionário, passamos para o próximo. Se não, utilizamos o último símbolo que tinhamos para I, criamos uma símbolo para I + c e recomeçamos com I = c.

Dessa forma, não é necessário transmitir o dicionário. No processe de decodificação, vamos descobrindo as palavras conforme decodificamos.
\begin{center}
\includegraphics[width=.9\linewidth]{/home/ieremies/org/.attach/bc/4bf780-600f-41d9-893b-752fce8e3170/_20230619_125245screenshot.png}
\end{center}
\item Comprimento de corrida
\label{sec:org2ace435}
Ou eu uso (onde\textsubscript{começa}, quanto\textsubscript{dura}) para cada corrida ou, começando de uma corrida preta, alterno em quanto dura a corrida.

Por árvore binária: eu divido a linha na metade e marco se as partes são inteiras brancas/pretas. Caso contrário, ramifico.
\item Planos de bits
\label{sec:org1a1f8d4}
Caso sua imagem não seja binária, não tema. Podemos utilizar os planos de bits que correspondem ao mapeamento de cada i-ésimo bit em cada valor.
\item Preditiva sem perdas
\label{sec:org27cbd60}
Redundância interpixel
Ao invez de salvar a variação entre os pixels, utilizamos uma função que tenta adivinhar qual o próximo pixel e que na verdade salvo o erro dessa função. Assim, na hora de decodificar, tentamos prever o pixel e adicionamos o erro salvo.
Quanto mais precisa for a predição, menor os valores de erro, menor o espaço.
\end{itemize}
\item Com perdas
\label{sec:orgec84ca4}
\begin{itemize}
\item Preditiva com perdas
\label{sec:org6775b38}
Mesmo processo do sem perdas, mas agora tendemos a jogar fora algumas informações do erro para que este não se acumule. A função quantizadora pode, por exemplo, considerar erros muito pequenos como zero. Para evitar que esse ``arredondamento'' acumule-se, utilizamo-no também no cálculo do erro.
\item Modulação delta
\label{sec:org5fdac09}
O preditor preve que o pixel será igual e o erro só pdoe ser +/- o delta
\item Modulação Còdigo de Pulso Diferencial
\label{sec:org9a9bb65}
Assume-se que o erro devido a quantização é irrelevante e utiliza-se um preditor mais sofisticado, de forma a minimizar o erro médio quadrático.
\item Transformada
\label{sec:org8055563}
Utiliza-se janelas pequenas de 8x8 ou 16x16 nas quais aplicamos trasnformadas como a de Fourrier ou discreta do cosseno para descobrir os coeficientes que descrevem aquela região. Podemos descartar os coeficientes que descrevem o menor número de informações a fim de reduzir o espaço de armazenamento e guardar o resto para ser decodificado.

Não podemos usar grandes janelas pois estas não possuem uniformidade o que causaria muitos coeficientes.
\end{itemize}
\end{itemize}
\subsubsection*{Padronização JPEG}
\label{sec:orgd86546f}
Ordenamos por zigue-zague para facilitar a codificação por entropia
Realizar as operações em matrizes maiores possui um custo computacional elevado quando tratamos de transformadas, mas, até um certo ponto, há vatagem já que o valor médio, chamado DC e cada janela, é muito similar em janelas vizinhas, o que torna muito útil técnicas preditivas.
\section*{Lista 2}
\label{sec:org503b4ec}
\subsection*{1. Estração de borda por operador morfológico}
\label{sec:org973b690}
\subsection*{2. Efeitos causados à representação quadtree após mudança em escala, tranlação ou rotação?}
\label{sec:org5262679}
\subsection*{3. Entropia / código de huffman}
\label{sec:org14e5cff}
\hyperref[sec:orge460ab4]{Compressão}
Entropia é dada pela soma \(- \sum p(s_i) log_2 p(s_i)\)
Código Huffman usa a redução de fontes na frequência que esses valores apreecem. COmbinando 95 com 169, depois o resultante com 21. 243 código 0, 21 código 10, 95 código 110 e 168 código 111.
\subsection*{4. Codificações de Huffman}
\label{sec:orgcf6358c}
\hyperref[sec:orge460ab4]{Compressão}
\begin{center}
\begin{tabular}{lrrrr}
simbol & prob & 1 & 2 & 3\\[0pt]
\hline
a & 0.55 & 0.55 & 0.55 & 0.55\\[0pt]
b & 0.15 & 0.15 & 0.30 & 0.45\\[0pt]
c & 0.15 & 0.15 & 0.15 & \\[0pt]
d & 0.10 & 0.15 &  & \\[0pt]
e & 0.05 &  &  & \\[0pt]
\end{tabular}
\end{center}

a = 0
b = 10
c = 110
d = 1110
e = 1111
média = 0.55 + 0.30 + 0.45 + 0.4 + 0.2 = 1.9

a = 0
b = 100
c = 101
d = 110
e = 111
média = 0.55 + 0.45 + 0.45 + 0.3 + 0.15 = 1.9

se eu fiz certo, eles possuem o mesmo comprimento médio
\subsection*{5. Construa dícionário e LZW}
\label{sec:orgfab910a}
\hyperref[sec:orge460ab4]{Compressão}
a b c bc cc ca ac cb bcc ccc cccc ccccc
1 2 2 0 2 3   4    9     10     5 9
b c c a c bc cc  ccc  cccc ca ccc

a b c
1 2 2 0 2 3 4 9 10 5 9
b
\subsection*{6. Cores subtrativos / aditivos}
\label{sec:orgd86bac7}
\hyperref[sec:org14fa4aa]{Cores}
Modelos de cores subtrativos são baseados na adição de filtros para remoção de frequências, como CMY utilizando em toners, apesar de nesse caso ser necessário adicionar o K de black para diminuir o uso de cores. Nesse modelo, preto é a presença de todos os filtros.

\subsection*{7. Vantagens da codificação aritmética para compressão}
\label{sec:org2056289}
\hyperref[sec:orge460ab4]{Compressão}
Ele é extremamente eficiente em tamanho da representação, apesar de conter problemas de precisão no desempenho da decodificação.
\subsection*{8. Redundância em compressão}
\label{sec:orgf5cca98}
\hyperref[sec:orge460ab4]{Compressão}
Redundância de coficação, interpixel e interpretação humana. A primeira se dá a um uso excessivo de pixels para representar os símbolos, a segunda pelo fato de pixels próximos terem valores próximos (podemos então usar compressões preditivas) e o terceiro se dá pelo fato do olho humana não conseguir distringuir todas as informações presentes e dar mais valores a algumas.
\subsection*{9. Vantagens e desvantagens de usar blocos de tamanhos diferentes no cálculo da transformada discreta do cosseno no JPEG}
\label{sec:org56b8366}
\hyperref[sec:orge460ab4]{Compressão}
Blocos maiores possuem mais informações mas são bem mais custosos de computar.
Blocos menores são mais fácieis de realizar a DCT mas abstraem menos da informação.
\subsection*{10. Vantagem da ordenação zig-zag do JPEG?}
\label{sec:org58b4dcd}
\hyperref[sec:orge460ab4]{Compressão}
Facilita a codificação por entropia dos coeficientes AC da transformada discreta do cosseno dentro das janelas 8x8.
\subsection*{11. Compressão com perda}
\label{sec:org443f3fb}
\hyperref[sec:orge460ab4]{Compressão}
Predição com perdas na qual fazemos uma quantização do erro em pról de reduzir o tamanho necessário deste.
Por transformada, na qual dividimos a imagem em partes, aplicamos uma transformada e descartamos coeficiêntes que menos adicionam informações.
\subsection*{12. Técnicas preditivas de compressão de imagens. Descreva a principal diferença entre técnicas preditivas sem e com perdas.}
\label{sec:org6bb8724}
\hyperref[sec:orge460ab4]{Compressão}
Técnicas preditivas sem perdas armazenam o valor exato do erro, enquanto técnicas com perda utilizam a quantização ou modularização do valor do erro para economizar espaço em detrimento de um pouco de qualidade.
\subsection*{13. Código de comprimento de corridas}
\label{sec:org475fc7b}
\hyperref[sec:orge460ab4]{Compressão}
3 4 4 4 2 1 2 1 3 \ldots{}
lembrar que transformamos a imagem num grande vetor 1D
\subsection*{14. Versão binária da unidade de textura, padrões locais binários.}
\label{sec:orgd0248db}
\hyperref[sec:org48121c0]{Textura}
Reduz o número de entradas no espectro de textura, o que permite uma representação mais sucinta.
\subsection*{15. Padrões locais binários demonstram ser invariantes a tranformações monotônicas aplicadas à imagem. Quais as vantagens?}
\label{sec:org88e99aa}
\hyperref[sec:org48121c0]{Textura}
O fato da transformação ser monotônica faz com que as comparações entre os valores dos pixels não mude, o que, pela definição de LBP, não modifica a representação. Assim, temos que, apesar de uma transformação alterar os valores da imagem, conseguimos manter a nossa representação do que é ainda a mesma estrutura de textura.

Podemos, sem medo de incubir o custo de recomputar o LBP fazer alterações e garantir que texturas que soferam apenas transformações monotônicas manteram sua unidade de textura.
\subsection*{16. Matriz de concorrência, momento angular, discrminação de texturas.}
\label{sec:orgb2f9b32}
\hyperref[sec:org48121c0]{Textura}
\begin{center}
\begin{tabular}{rrrrr}
x & 0 & 1 & 2 & 3\\[0pt]
0 &  &  &  & \\[0pt]
1 &  &  & 6 & \\[0pt]
2 &  & 6 &  & \\[0pt]
3 &  &  &  & \\[0pt]
\end{tabular}
\end{center}

Momento angular = 1/12\textsuperscript{2} + 1/12\textsuperscript{2} = 1/72

\begin{center}
\begin{tabular}{rrrrr}
x & 0 & 1 & 2 & 3\\[0pt]
0 &  & 1 &  & \\[0pt]
1 & 1 &  & 2 & \\[0pt]
2 &  & 2 &  & 3\\[0pt]
3 &  &  & 3 & \\[0pt]
\end{tabular}
\end{center}

Segundo Momento Angular = (2 + 8 + 18)/12\textsuperscript{2} = 14/72

Essa medida pode ser usada sim, SMA maiores, representam níveis de energia maiores e portanto texturas ásperas

\textbf{SEGUNDO MOMENTO ANGULAR ALTO -> ÁSPERA}
\subsection*{17. Rotações 2d são aditivas}
\label{sec:orgd887ed8}
Ou seja, rotação por \(\alpha_1\) e posterior rotação por \(\alpha_2\) é igual a rotação por \(\alpha = \alpha_1 + \alpha_2\).

Só expandir as definições de rotação como x = xcos - siny e y = xsin + ycos. No final tem que fazer a regrinha de trigonometria.
\subsection*{18. Escalas 2d são multiplicativas}
\label{sec:org2cd53dc}
São multiplicatias pq na conta dos fatores fica uma multiplicação entre eles vezes o valor original
\subsection*{19. Mostre que a rotação e escala são comutativas se os fatores de escala S\textsubscript{x} = S\textsubscript{y}}
\label{sec:org2826976}
Mesma ideia das anteriores.
\subsection*{20. Transformada afim}
\label{sec:org9a60db3}
As transformadas afins generalizam transformações como rotação, translação, escala enquanto mantém o paralelismo e as proporções entre volumes, áreas e comprimentos entre objetos da imagem. Podem ser representadas na forma matricial com coordenada homogênea de forma que a útlima linha seja 0 0 0 1.
\subsection*{21. Projeção ortográfica e projeção perspectiva.}
\label{sec:orgc43ae95}
\hyperref[sec:org0e52fa1]{Registro}
Projeção ortográfica possui um centro de projeção no infinito, mantém as retas paralelas e as dimensões intactas.
Já a projeção de perspectiva modifica a dimensão dos objetos baseado em suas distâncias ao foco da lente (distância focal)
\subsection*{23. Coordenadas homogêneas para representação de transformações geométricas}
\label{sec:org2d2c35b}
\hyperref[sec:org0e52fa1]{Registro}
Permite que modelemos as transformações via matrizes e utilizemos as operações matriciais para combinar transformadas.
\subsection*{24. Descrve três técnicas de registro de imagem.}
\label{sec:org4825e4b}
\hyperref[sec:org0e52fa1]{Registro}
\end{document}