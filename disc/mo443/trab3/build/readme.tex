% Created 2023-05-26 Fri 14:28
% Intended LaTeX compiler: pdflatex
\documentclass[twocolumn, 10pt]{article}
\usepackage[utf8]{inputenc}
\usepackage[T1]{fontenc}
\usepackage{graphicx}
\usepackage{longtable}
\usepackage{wrapfig}
\usepackage{rotating}
\usepackage[normalem]{ulem}
\usepackage{amsmath}
\usepackage{amssymb}
\usepackage{capt-of}
\usepackage{hyperref}
\usepackage{todonotes}
\usepackage[brazil, portuges]{babel}
\usepackage{amsthm}
\author{Ieremies Vieira da Fonseca Romero}
\date{}
\title{Trabalho 3\\\medskip
\large Processamento Digital de Imagem}
\hypersetup{
 pdfauthor={Ieremies Vieira da Fonseca Romero},
 pdftitle={Trabalho 3},
 pdfkeywords={},
 pdfsubject={},
 pdfcreator={Emacs 28.2 (Org mode 9.6.1)}, 
 pdflang={Portuges}}
\usepackage[date=year]{biblatex}
\addbibresource{~/arq/bib.bib}
\begin{document}

\maketitle
Conferir se zero é preto usando uma iamgem pequena com meu nome ou algo assim

\begin{verbatim}
import cv2
import numpy as np

img = cv2.imread('bitmap.pbm', cv2.IMREAD_UNCHANGED)
img[img == 0] = 1   # all black pixels to 1
img[img == 255] = 0 # all whites pixels to 0
\end{verbatim}
Agora nós temos uma imagem representada por zeros e uns onde o texto é demarcado pelos pixels com valor de \(1\).


\begin{verbatim}
def save(img):
    # turn back to gray scale
    img[img == 0] = 255
    img[img == 1] = 0
    cv2.imwrite('out.png', img)

save(img)
\end{verbatim}
\end{document}