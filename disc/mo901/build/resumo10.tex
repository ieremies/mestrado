% Created 2023-05-26 Fri 14:06
% Intended LaTeX compiler: pdflatex
\documentclass[11pt]{article}
\usepackage[utf8]{inputenc}
\usepackage[T1]{fontenc}
\usepackage{graphicx}
\usepackage{longtable}
\usepackage{wrapfig}
\usepackage{rotating}
\usepackage[normalem]{ulem}
\usepackage{amsmath}
\usepackage{amssymb}
\usepackage{capt-of}
\usepackage{hyperref}
\usepackage{todonotes}
\usepackage[brazil, portuges]{babel}
\usepackage{amsthm}
\author{Ieremies Vieira da Fonseca Romero}
\date{}
\title{Disseminação de Dados Pessoais Vitais Para Apoio às Tomadas de Decisão em Situações Emergenciais em Ambientes Externos\\\medskip
\large Seminário 10}
\hypersetup{
 pdfauthor={Ieremies Vieira da Fonseca Romero},
 pdftitle={Disseminação de Dados Pessoais Vitais Para Apoio às Tomadas de Decisão em Situações Emergenciais em Ambientes Externos},
 pdfkeywords={},
 pdfsubject={},
 pdfcreator={Emacs 28.2 (Org mode 9.6.1)}, 
 pdflang={Portuges}}
\usepackage[date=year]{biblatex}
\addbibresource{~/arq/bib.bib}
\begin{document}

\maketitle
A palestra abordou a questão da disseminação de dados pessoais sensíveis em situações emergenciais em ambientes externos, destacando a importância de garantir a confidencialidade desses dados e o controle de acesso a eles. Foi ressaltado que, mesmo em ambientes urbanos, não podemos presumir tranquilidade e harmonia na infraestrutura de rede.

Em eventos críticos, como quedas ou acidentes, o compartilhamento de informações e dados pessoais sensíveis se torna crucial para o apoio às tomadas de decisão. No entanto, a segurança desses dados e o controle de acesso podem ser comprometidos em situações críticas, aumentando a dependência de pessoas desconhecidas.

A palestra destacou a importância do tempo nessas situações, enfatizando que cada minuto em parada respiratória ou cardíaca reduz consideravelmente as chances de sobrevivência. Nesse contexto, foi apresentada uma solução chamada STEALTH (Social Thrust Health Information Access Control), baseada em conceitos como confiança social, controle de acesso e interesse comum.

O STEALTH permite que pessoas com competências e interesses sociais relevantes tenham acesso aos dados de maneira controlada. Por exemplo, um médico poderia ter acesso aos dados em casos que requerem assistência médica. Essa tecnologia visa criar uma rede de apoio baseada nas competências individuais, de forma que, durante um evento crítico, os agentes mais próximos e competentes sejam acionados e os dados sejam compartilhados de acordo com a competência de cada indivíduo.

Em resumo, a palestra discutiu a importância da disseminação de dados pessoais vitais em situações emergenciais, apresentando o conceito do STEALTH como uma solução que combina confidencialidade, controle de acesso e compartilhamento de informações com base nas competências individuais.
\end{document}