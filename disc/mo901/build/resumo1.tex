% Created 2023-03-17 Fri 19:07
% Intended LaTeX compiler: pdflatex
\documentclass[11pt]{article}
\usepackage[utf8]{inputenc}
\usepackage[T1]{fontenc}
\usepackage{graphicx}
\usepackage{longtable}
\usepackage{wrapfig}
\usepackage{rotating}
\usepackage[normalem]{ulem}
\usepackage{amsmath}
\usepackage{amssymb}
\usepackage{capt-of}
\usepackage{hyperref}
\usepackage{todonotes}
\usepackage[brazil, portuges]{babel}
\usepackage{amsthm}
\author{Ieremies Vieira da Fonseca Romero}
\date{}
\title{Transformação Digital – Computação Escalar \& Dados Digitais Intensivos\\\medskip
\large Seminário 1}
\hypersetup{
 pdfauthor={Ieremies Vieira da Fonseca Romero},
 pdftitle={Transformação Digital – Computação Escalar \& Dados Digitais Intensivos},
 pdfkeywords={},
 pdfsubject={},
 pdfcreator={Emacs 28.2 (Org mode 9.6.1)}, 
 pdflang={Portuges}}
\usepackage[date=year]{biblatex}
\addbibresource{~/arq/bib.bib}
\begin{document}

\maketitle
A computação escalável com uso intensivo de dados (DISC) é uma abordagem centrada em dados digitais que utiliza modelos de programação de alto nível e clusters de hardware para processamento. A estratégia de downsizing, que tentava diminuir o tamanho dos computadores, foi substituída pela abordagem de rightsizing, que permite a utilização de diferentes níveis de computação (edge, fog e cloud) para executar tarefas específicas de forma mais eficiente.

No contexto da DISC, os dados são coletados na borda, limpos e organizados para serem enviados para a “fog”, que agrega os dados e envia para processamento na nuvem. A análise desses dados permite a extração de informações valiosas para o conhecimento.

A DISC é aplicada em cenários de workflows de dados, paralelização massiva, logística de dados, processamento de stream, entre outros. É importante considerar a escalabilidade de entrada e saída de dados, a capacidade de armazenamento e o desempenho de acesso aos dados. Além disso, é necessário utilizar ferramentas específicas para cada uma dessas tarefas.

A simulação apropriada, de pós-processamento, in-situ e in-transit também é uma parte importante da abordagem DISC. Em resumo, a DISC é uma abordagem centrada em dados que requer uma organização e armazenamento eficientes dos dados, além de ferramentas específicas para processamento e simulação (pós-processamento, in-situ e in-transit).
\end{document}