% Created 2023-03-17 Fri 19:08
% Intended LaTeX compiler: pdflatex
\documentclass[11pt]{article}
\usepackage[utf8]{inputenc}
\usepackage[T1]{fontenc}
\usepackage{graphicx}
\usepackage{longtable}
\usepackage{wrapfig}
\usepackage{rotating}
\usepackage[normalem]{ulem}
\usepackage{amsmath}
\usepackage{amssymb}
\usepackage{capt-of}
\usepackage{hyperref}
\usepackage{todonotes}
\usepackage[brazil, portuges]{babel}
\usepackage{amsthm}
\author{Ieremies Vieira da Fonseca Romero}
\date{}
\title{Exploiting the Computing Continuum with self-distributing services\\\medskip
\large Seminário 2}
\hypersetup{
 pdfauthor={Ieremies Vieira da Fonseca Romero},
 pdftitle={Exploiting the Computing Continuum with self-distributing services},
 pdfkeywords={},
 pdfsubject={},
 pdfcreator={Emacs 28.2 (Org mode 9.6.1)}, 
 pdflang={Portuges}}
\usepackage[date=year]{biblatex}
\addbibresource{~/arq/bib.bib}
\begin{document}

\maketitle
Os sistemas contemporâneos apresentam elevada complexidade devido à heterogeneidade (de hardware, sistema operacional, linguagem de programação, protocolos etc.) e volatilidade (constante mudança em diferentes níveis, adição e remoção de máquinas, mudanças no uso da rede e padrões de uso).

Atualmente, a estrutura de computação distribuída estabelece uma hierárquia que inclui dispositivos na borda (PCs, celulares e outros dispositivos com recursos computacionais limitados), nós de borda (que agregam informações de dispositivos de borda e enviam informações para a nuvem) e nuvem (camada mais distante).

Computação contínua visa integrar essas camadas de forma a reduzir as barreiras.
Serviços como contêineres e Kubernetes permitem a mobilidade do código nas diferentes camadas.
Para quebrar um sistema e movê-lo de camada, é necessário primeiro transformá-lo de uma aplicação monolítica em microsserviços autônomos sem estado.

Os sistemas de software emergentes são sistemas autônomos e autoadaptativos que utilizam algoritmos de aprendizado por reforço para gerenciar suas componentes. Esse agente inteligente pode trocar as componentes em tempo de execução e determinar a melhor configuração com base no impacto nos recursos e funcionalidades. Esse conceito também pode ser usado para determinar a colocação das componentes em diferentes camadas ou clusters de rede. Embora isso possa aumentar os requisitos de rede, o off-loading de carga computacional pode ser benéfico em muitos casos.
\end{document}