% Created 2023-03-31 sex 19:48
% Intended LaTeX compiler: pdflatex
\documentclass[11pt]{article}
\usepackage[utf8]{inputenc}
\usepackage[T1]{fontenc}
\usepackage{graphicx}
\usepackage{longtable}
\usepackage{wrapfig}
\usepackage{rotating}
\usepackage[normalem]{ulem}
\usepackage{amsmath}
\usepackage{amssymb}
\usepackage{capt-of}
\usepackage{hyperref}
\usepackage[outpudir=build]{minted}
\usepackage[outpudir=build]{minted}
\usepackage{todonotes}
\usepackage[brazil, portuges]{babel}
\usepackage{amsthm}
\author{Ieremies Vieira da Fonseca Romero}
\date{}
\title{Aprendizagem federada em ambientes de cidades inteligentes\\\medskip
\large Seminário 3}
\hypersetup{
 pdfauthor={Ieremies Vieira da Fonseca Romero},
 pdftitle={Aprendizagem federada em ambientes de cidades inteligentes},
 pdfkeywords={},
 pdfsubject={},
 pdfcreator={Emacs 28.2 (Org mode 9.6.1)}, 
 pdflang={Portuges}}
\usepackage[date=year]{biblatex}
\addbibresource{~/arq/bib.bib}
\begin{document}

\maketitle
Na palestra ``Aprendizagem federada em ambientes de cidades inteligentes'', foi discutido que a abordagem mais comum em relação aos dados é centralizá-los. No entanto, essa prática traz problemas, principalmente relacionados à privacidade. Portanto, uma alternativa seria manter os dados na borda, no dispositivo que os coletou, e utilizar o aprendizado federado para treinar modelos conjuntamente com outros dispositivos.

No aprendizado federado, um conjunto de dados locais é utilizado para melhorar o modelo de forma personalizada, e o dispositivo de borda recebe o modelo geral, utiliza seus próprios dados para melhorar o modelo localmente e, em seguida, devolve os parâmetros. O agregador do aprendizado federado aplica um algoritmo de agregação cujo objetivo é juntar os ``conhecimentos'' adquiridos pelos dispositivos de borda.

Além da questão de privacidade, o aprendizado federado apresenta outras vantagens, como a capacidade da rede, a latência mais baixa e a escalabilidade, principalmente em situações de alta demanda. Para a implementação, existem diversas ferramentas disponíveis, como TensorFlow Federated (TFF), FATE, Sherpa.ai, DataSHIELD e Flower.

Entre os algoritmos de agregação, podemos citar o Google Federated Average, o Federated SGD, o Fault Tolerant Federated Average, o Q-Federated Average, o Federated Optimization e o FedSim, bons para diferentes tipos de dados e situações.

No entanto, os desafios relacionados ao aprendizado federado ainda são significativos, como a privacidade, a insuficiência de dados, a heterogeneidade estatística, a latência e a mobilidade. São necessárias mais pesquisas e avanços na área para superar esses desafios e potencializar o uso do aprendizado federado.

Po fim, o palestrante contou suas experiências trabalhando em um projeto financiado pela União Europeia que utiliza a cidade de Aveiro, em Portugal, como experimento nessa tecnologia.
\end{document}