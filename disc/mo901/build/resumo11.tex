% Created 2023-06-16 Fri 20:05
% Intended LaTeX compiler: pdflatex
\documentclass[11pt]{article}
\usepackage[utf8]{inputenc}
\usepackage[T1]{fontenc}
\usepackage{graphicx}
\usepackage{longtable}
\usepackage{wrapfig}
\usepackage{rotating}
\usepackage[normalem]{ulem}
\usepackage{amsmath}
\usepackage{amssymb}
\usepackage{capt-of}
\usepackage{hyperref}
\usepackage{todonotes}
\usepackage[brazil, american]{babel}
\usepackage{amsthm}
\author{Ieremies Vieira da Fonseca Romero}
\date{}
\title{Internet das Coisas Móveis (IoMT) e Coordenação de UAVs para aplicações do meio ambiente\\\medskip
\large Seminário 11}
\hypersetup{
 pdfauthor={Ieremies Vieira da Fonseca Romero},
 pdftitle={Internet das Coisas Móveis (IoMT) e Coordenação de UAVs para aplicações do meio ambiente},
 pdfkeywords={},
 pdfsubject={},
 pdfcreator={Emacs 28.2 (Org mode 9.6.1)}, 
 pdflang={English}}
\usepackage[date=year]{biblatex}

\begin{document}

\maketitle
A palestra abordou a importância da conectividade sem fio e intermitente na coordenação e orquestração de objetos móveis, produtos ou veículos para o funcionamento eficiente de nossas vidas, economia e meio ambiente.

A Internet das Coisas Móveis (IoMT) é caracterizada pelo movimento relativo entre coletores, hubs e dispositivos inteligentes (IoT) e pela conectividade intermitente proporcionada pela alcançabilidade do rádio WPAN. Além disso, o número de dispositivos IoT atendidos por cada coletor pode variar.

O projeto iniciou ao reconhecer que os smartphones são excelentes hubs para a coleta de dados desses dispositivos móveis. Os pesquisadores também colaboraram com a força aérea em um projeto chamado GrADyS, no qual coordenaram quadcópteros para transportar dados entre redes mesh.

O protocolo DADCA, desenvolvido para dispositivos IoT estacionários, adota uma heurística flexível e simples. Utiliza um padrão de movimento em zig-zag para cobrir uma área maior e transferir dados entre os drones e as estações de base.

Tais abordagens pussuem o potencial de melhorar a coordenação e a eficiência na coleta de dados, crucial para aplicações ambientais. Ao utilizar a Internet das Coisas Móveis e a coordenação de UAVs (veículos aéreos não tripulados), podemos obter informações valiosas, o que contribui para a preservação e o monitoramento do meio ambiente.
\end{document}