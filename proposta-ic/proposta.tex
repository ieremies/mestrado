% Created 2022-09-13 ter 12:58
% Intended LaTeX compiler: pdflatex
\documentclass[11pt]{article}
\usepackage[utf8]{inputenc}
\usepackage[T1]{fontenc}
\usepackage{graphicx}
\usepackage{longtable}
\usepackage{wrapfig}
\usepackage{rotating}
\usepackage[normalem]{ulem}
\usepackage{amsmath}
\usepackage{amssymb}
\usepackage{capt-of}
\usepackage{hyperref}
\usepackage{todonotes}
\usepackage[portuges]{babel}
\usepackage{amsthm}
\usepackage[a4paper, total={6in, 8in}]{geometry}
\usepackage{multicol}
\author{Ieremies Vieira da Fonseca Romero}
\date{}
\title{Problema de Dominação Romana}
\hypersetup{
 pdfauthor={Ieremies Vieira da Fonseca Romero},
 pdftitle={Problema de Dominação Romana},
 pdfkeywords={},
 pdfsubject={},
 pdfcreator={Emacs 28.1 (Org mode 9.6)}, 
 pdflang={Portuges}}
\usepackage{biblatex}
\addbibresource{~/arq/bib.bib}
\begin{document}

\maketitle
\begin{abstract}
A Dominação Romana é um problema proposto por \textcite{Stewart1999DefendRomanEmpire} em que desejamos defender o império romano dispondo de um certo número de legiões.
Para isso, cada cidade deve ser assegurada de forma que uma legião alocada ou seja vizinha de outra que possua duas legiões.
Assim, deseja-se minimizar a quantidade de legiões distribuídas sem abdicar da segurança do império.

Neste projeto, utilizaremos a técnica de Programação Linear Inteira (PLI) para modelar o problema.
Objetivamos propor novos modelos de PLI para dominação romana e suas variantes, explorando técnicas como branch-cut, branch-price e ferramental moderno de PLI.
\end{abstract}

\section{Introdução}
\label{sec:org3375f23}
\subsection{Motivação histórica}
\label{sec:org316761a}
Durante a Segunda Guerra Mundial, General Douglas MacArthur propôs uma estratégia de movimentação que consistia em avançar suas tropas de uma ilha para outra apenas quando ele poderia deixar para trás um número suficiente de tropas (\autocite{Stewart1999DefendRomanEmpire}).
Ele não foi o primeiro a utilizar dessa estratégia: segundo \textcite{Stewart1999DefendRomanEmpire}, referências históricas apontam que o Imperador Constantino, no quarto século A.C., aplicou estratégia similar para defender o Império Romano de invasões dos povos ditos "bárbaros".\todo{aqui têm refs de refs}

Para exemplificar o seu uso, considere o mapa do Império Romano simplificado na Figura \ref{fig:mapa}.
Nesse exemplo, o imperador possui 4 legiões para serem distribuídas pelo território e ele deseja fazê-lo de forma que todas as cidades sejam consideradas seguras.
Uma região é dita segura, ou coberta, se há uma legião em seu território ou se está conectada a outra região com duas legiões.

\begin{figure}[htbp]
\centering
\includegraphics[scale=0.3]{/home/ieremies/org/.attach/7f/0c2fd1-ace0-4f61-b2ca-58a004a599d0/_20220817_122014screenshot.png}
\caption{\label{fig:mapa}Representação do mapa do Império Romano usada como ilustração do problem, inspirada de \textcite{Stewart1999DefendRomanEmpire}.}
\end{figure}

As aplicações vão além do universo militar: ao alocar estações de serviços de emergência, uma lógica similar à "segurança", acima descrita, é muito útil \autocite{GhaffariHadigheh2019Romandominationproblem}. Além disso, redes sem fio "ad hoc", \autocite{Wu2000Dominationitsapplications} \todo{completar}.


\subsection{Modelo matemático}
\label{sec:orgcac6fc5}
Para um grafo \(G = (V, E)\), dizemos que a \textbf{vizinhança aberta} \(N(v)\) de um vértice \(v\) é definida como o conjunto de vértices adjacentes a \(v\) em \(G\), ou seja, \(N(v) = \{u|(u, v) \in E\}\).
Similarmente, dizemos que a \textbf{vizinhança fechada} \(N[v]\) de um vértice \(v\) é a vizinhança aberta incluindo o próprio \(v\), ou seja, \(N[v] = N(v) \cup \{v\}\).
Para um conjunto de vértices \(s\), a vizinhança aberta desse conjunto é a união das vizinhanças abertas de cada um dos seus vértices (o respectivo pode ser dito para a vizinhança fechada).
Um \textbf{conjunto dominante} de um grafo \(G\) é um subconjunto de vértices \(D\) tal que a vizinhança fechada de \(D\) é o próprio conjunto \(V\) .
Por sua vez, o \textbf{número de dominação} de um grafo \(G\), dito \(\gamma(G)\), é a cardinalidade do menor conjunto dominante do grafo \(G\).

O problema de dominação romana é definido em um grafo \(G = (V, E)\) simples, finito e não-direcionado, no qual cada vértice representa uma cidade ou região do império e as arestas são as conexões entre elas \autocite{Cockayne2004Romandominationgraphs}.
Diremos que uma \textbf{função de dominação romana} é uma função \(f : V \to z{0, 1, 2\}\) na qual \(f(v)\) indica a quantidade de legiões naquela região,de forma que, para qualquer v tal que \(f(v) = 0\), deve existir \(u\) vizinho a \(v\) cujo \(f(u) = 2\).
Definimos o \textbf{número de dominação romana} total de um grafo \(G\) como o menor valor \(f(v), \forall v \in V\), tal que f é uma função de dominação romana do grafo G.

\subsection{Revisão bibliográfica}
\label{sec:orgb4834d2}
Após a descrição inicial do problema, \textcite{ReVelle2000DefendensImperiumRomanum} apresentaram o desenvolvimento inicial em teoria de grafos.
Além disso, \textcite{Cockayne2004Romandominationgraphs} apresentaram alguns resultados importantes de teoria de grafos sobre o problema, como limitantes e propriedades da função de dominação romana, os quais foram estendidos e aprimorados por \textcite{Xing2006noteRomandomination}, \textcite{Favaron2009Romandominationnumber}, \textcite{Mobaraky2008BoundsRomanDomination}.
Algumas classes especiais de grafos podem ser resolvidas em tempo linear, mas, no caso geral, o problema é NP-difícil (\autocite{Dreyer2000Applicationsvariationsdomination,Klobucar2014SomeresultsRoman,Shang2007RomanDominationProblem}).

\textcite{Ivanovic2016Improvedmixedinteger} utilizaram \emph{Variable Neightborhood Search} (VNS) no mesmo problema, obtendo resultados interessantes para as instâncias propostas por \textcite{Curro2014RomanDominationProblem}.
Essa meta-heurística parte da ideia de que soluções ótimas são encontradas "próximas" de boas soluções, assim utilizando busca local e algumas técnicas, como perturbação,\todo{tá ruim isso} para escapar de mínimos locais e intensificar a procura.

Já \textcite{Khandelwal2021RomanDominationGraphs} utilizaram algoritmos genéticos no problema de dominação romana, uma ideia que toma de inspiração da evolução das espécies observadas na natureza.
Partindo de um conjunto de soluções, realizamos "cruzamentos" das melhores para produzir novas gerações.
A cada uma, induzimos "mutações" aleatórias que alteram certos pontos das soluções, espelhando a realidade e tentando evitar cair em mínimos locais.

\section{Metodologia}
\label{sec:orge63a0a7}

\subsection{Programação linear}
\label{sec:org42ce7bc}
\begin{itemize}
\item Definição de PL
\end{itemize}
Programação Linear é uma técnica de optimização de problemas a partir da modelagem dos mesmos por meio de programas lineares.
Nestes, definimos uma função objetivo, a qual queremos maximizar ou minimizar com suas variáveis sujeitas a um conjunto de restrições lineares (equações ou inequações lineares). \todo{citar um bom livro} Todo programa linear pode ser escrito em sua forma canônica:
\begin{align*}
\text{maximize }  &cx \\
\text{sujeito a } &Ax \leq b \\
                  &x \in \mathbb{R}_+
\end{align*}

Perceba que maximizar uma função é o mesmo que minimizar a mesma com sinal oposto.

Para resolver esse tipo de programa, conhecemos o algoritmo \emph{simplex} que, apesar de ser exponencial \todo{impreciso}, no caso médio possui comportamento polinomial.

Pega o PL na forma padrão, adiciona as variáveis de folga, ou seja, a diferença das inequações. Estas variáveis adicionadas são chamadas de básicas enquanto as demais de não básicas. Uma vez feito isso, iteramos :
\begin{enumerate}
\item Acha uma solução viável
\item Pivota até achar uma solução ótima
\begin{enumerate}
\item Selecione a variável não-básica com maior coeficiente positivo.
\item Aumente seu valor o máximo possível
\item Ache a restrição mais justa, que limita o passo anterior.
\item Inverte as posições da variávei não-básica escolhida no passo 1 com a variável básica da restrição do passo 3.
\item Repita até não existir nenhuma variável que satisfaça o passo 1.
\end{enumerate}
\end{enumerate}

Para alguns problemas, como o de dominação romana, não faz sentido falar em soluções fracionárias, afinal não conseguimos "alocar meia legião". \todo{cite um bom livro}
Para isso, restringimos as variáveis aos inteiros, fazendo assim um \textbf{Programa Linear Inteiro}.

O que a princípio pode parecer uma pequena alteração, torna o problema computacionalmente ainda mais complexo.
\begin{itemize}
\item atual modelo para dominação romana
\end{itemize}
\begin{multicols}{2}
$$
x_i=\left\{\begin{array}{ll}
1, & f(i) \geqslant 1 \\
0, & \text { otherwise }
\end{array} \quad y_i= \begin{cases}1, & f(i)=2 \\
0, & \text { otherwise. }\end{cases}\right.
$$
\begin{alignat}{4}
(\mathcal{RR}) \quad & \omit\rlap{minimize  $\displaystyle \sum_{i \in V} x_i+\sum_{i \in V} y_i$} \\
& \mbox{sujeito a}&& \quad & x_i+\sum_{j \in N_i} y_j & \geq 1  & \quad & i \in V \\
&                 &&       & y_i                    & \leq x_i &        & i \in V \\
&                 &&       & x_i, y_i               & \in\{0,1\} &      & i \in V
\end{alignat}
$$
x_i=\left\{\begin{array}{ll}
1, & f(i)=1 \\
0, & \text { otherwise }
\end{array} \quad y_i= \begin{cases}1, & f(i)=2 \\
0, & \text { otherwise. }\end{cases}\right.
$$
\begin{alignat}{4}
(\mathcal{BVV}) \quad & \omit\rlap{minimize  $\displaystyle \sum_{i \in V} x_i+2\sum_{i \in V} y_i$} \\
& \mbox{sujeito a}&& \quad & x_i+y_i+\sum_{j \in N_i} y_j & \geq 1  & \quad & i \in V \\
&                 &&       & x_i + y_i                    & \leq x_i &        & i \in V \\
&                 &&       & x_i, y_i               & \in\{0,1\} &      & i \in V
\end{alignat}
\end{multicols}
\begin{itemize}
\item citar a ideia do "ferramental moderno de PLI"
\end{itemize}

\section{Objetivos}
\label{sec:orgff6a726}
Os algoritmos e modelos propostos serão comparados com as instâncias presentes na literatura, como em \textcite{Curro2014RomanDominationProblem} e, se necessário novas instâncias poderão ser geradas.

Os resultados dos experimentos computacionais serão comparados utilizando técnicas como \emph{Performance Profile} demonstrado por \textcite{Dolan2002Benchmarkingoptimizationsoftware}.

\printbibliography
\end{document}
