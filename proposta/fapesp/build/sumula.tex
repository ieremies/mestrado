% Created 2023-03-07 Tue 20:24
% Intended LaTeX compiler: pdflatex
\documentclass[11pt]{article}
\usepackage[utf8]{inputenc}
\usepackage[T1]{fontenc}
\usepackage{graphicx}
\usepackage{longtable}
\usepackage{wrapfig}
\usepackage{rotating}
\usepackage[normalem]{ulem}
\usepackage{amsmath}
\usepackage{amssymb}
\usepackage{capt-of}
\usepackage{hyperref}
\usepackage{todonotes}
\usepackage[portuges]{babel}
\usepackage{amsthm}
\usepackage[a4paper, total={6in, 8in}]{geometry}
\author{Ieremies Vieira da Fonseca Romero}
\date{}
\title{Súmula Curricular\\\medskip
\large Programação Linear Inteira aplicada a problemas de Coloração em Grafos}
\hypersetup{
 pdfauthor={Ieremies Vieira da Fonseca Romero},
 pdftitle={Súmula Curricular},
 pdfkeywords={},
 pdfsubject={},
 pdfcreator={Emacs 28.2 (Org mode 9.6.1)}, 
 pdflang={Portuges}}
\usepackage{biblatex}
\addbibresource{~/arq/bib.bib}
\begin{document}

\maketitle

\section*{Formação}
\label{sec:orgd294247}
\textbf{Bacharelado em Ciência da Computação}. Instituto de Computação da Universidade Estadual de Campinas - UNICAMP.
Início: Março de 2018.
Conclusão: Dezembro de 2022.

\section*{Histórico Profissional}
\label{sec:orga2ea44e}
\begin{description}
\item[{Professor de programação para ensino médio}] (jan, 2022 - atual) \\ Colégio e pré-vestibular Elite Campinas.
\item[{Coordenador de unidade no UNICAMP de Portas Abertas}] (ago, 2022) \\ Instituto de Computação da UNICAMP.
\item[{Representante discente na Comissão de Graduação}] (abr, 2022 - dez, 2022) \\ Instituto de Computação da UNICAMP.
\item[{Representante discente na Congregação}] (jul, 2021 - abr, 2022) \\ Instituto de Computação da UNICAMP.
\item[{Estagiário em desenvolvimento de software}] (mar, 2022 - abr, 2022) \\ SF-Labs, Visio.
\item[{Bolsista de Iniciação Científica}] (ago, 2019 - jul, 2020) \\ Fundação de Amparo a Pesquisa do Estado de São Paulo
\item[{Coordenador de cursinho popular}] (ago, 2018 - abr, 2019) \\ Projeto Além da Escola
\item[{Professor voluntário de programação em cursinho popular}] (mar, 2018 - abr, 2019) \\ Projeto Além da Escola.
\end{description}
\section*{Resultados}
\label{sec:orgd976e2b}
\begin{itemize}
\item da Silva FJ, Miyazawa FK, Romero IV, Schouery R. Tight Bounds for the Price of Anarchy and Stability in Sequential Transportation Games. arXiv preprint arXiv:2007.08726. 2020 Jul 17.
\end{itemize}
\section*{Financiamentos à pesquisa}
\label{sec:orgd903a3a}
Recebi recursos correspondentes a bolsa de Iniciação Científica da Fundação de Amparo à Pesquisa do Estado de São Paulo no projeto \textbf{``Aplicações de Teoria dos Jogos Algorítmica a Problemas de Transporte''} (processo 2019/14492-9) entre agosto de 2019 e julho de 2020, o qual recebeu conceito máximo.
\section*{Indicadores quantitativos}
\label{sec:org3ef57ea}
Não se aplica
\section*{Links}
\label{sec:orgeb45a3b}
\begin{description}
\item[{ORCID}] orcid.org/0000-0002-5801-3774
\end{description}
\section*{Outras informações}
\label{sec:orgd8a9cb8}
Como parte de minha formação acadêmica, tive a oportunidade de ministrar dois workshops para alunos de graduação, um de git e outro de python, os quais foram bem recebidos pelos estudantes.
Além disso, nos últimos anos, tenho me dedicado também ao ensino de programação para jovens no ensino médio e avançando meus conhecimentos com disciplinas extracurriculares na área.

\subsection*{Premiações}
\label{sec:orgeb3822b}
\begin{itemize}
\item Primeiro lugar da segunda turma de formandos de 2022, na qual me graduei.
\item Destaque Instituto de Computação - UNICAMP e Sociedade Brasileira de Computação pelo desempenho na graduação.
\item Primeiro lugar do vestibular em Ciência da Computação - UNICAMP no ano de $2018$.
\end{itemize}
\end{document}
