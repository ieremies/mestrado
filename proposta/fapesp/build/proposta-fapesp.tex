% Created 2023-06-21 Wed 13:51
% Intended LaTeX compiler: pdflatex
\documentclass[11pt]{article}
\usepackage[utf8]{inputenc}
\usepackage[T1]{fontenc}
\usepackage{graphicx}
\usepackage{longtable}
\usepackage{wrapfig}
\usepackage{rotating}
\usepackage[normalem]{ulem}
\usepackage{amsmath}
\usepackage{amssymb}
\usepackage{capt-of}
\usepackage{hyperref}
\usepackage{todonotes}
\usepackage[brazil, brazilian]{babel}
\usepackage{amsthm}
\usepackage{setspace}
\doublespacing
\newtheorem{prep}{Preposição}[section]
\usepackage[a4paper, total={6in, 8in}]{geometry}
\usepackage[backend=biber,ittitles,justify,indent,sccite,giveninits,date=year]{biblatex}
\addbibresource{~/arq/bib.bib}
\renewbibmacro{in:}{\ifentrytype{article}{}{\printtext{\bibstring{in}\intitlepunct}}}
\author{Ieremies Vieira da Fonseca Romero}
\date{}
\title{Programação Linear Inteira aplicada a problemas de Coloração em Grafos}
\hypersetup{
 pdfauthor={Ieremies Vieira da Fonseca Romero},
 pdftitle={Programação Linear Inteira aplicada a problemas de Coloração em Grafos},
 pdfkeywords={},
 pdfsubject={},
 pdfcreator={Emacs 28.2 (Org mode 9.6.1)}, 
 pdflang={Portuges}}
\begin{document}

\maketitle
\section{Objetivos}
\label{sec:orgb064508}
Recentemente, houve avanços significativos na abordagem de branch-and-price, resultando em vários trabalhos publicados para outros problemas que utilizam e melhoram essa técnica.
Um exemplo notável é o trabalho de \textcite{Lima2022Exactsolutionnetwork}, que apresenta técnicas fortes que melhoram o estado da arte para problemas como o Problema de Empacotamento.

Os autores propõem um modelo baseado em fluxo de arcos para auxiliar em algoritmos de geração de colunas.
Eles também comentam que qualquer problema de cobertura de conjunto consegue ser transformado em um problema de fluxo, o que indica que suas técnicas podem ser relevantes para o nosso problema.
Além disso, eles utilizam \textbf{fixação de variáveis}, técnica na qual é possível provar que algumas variáveis nunca poderão entrar no modelo de \emph{branch-and-pricing} e tornar a solução melhor.
A dificuldade reside em encontrar uma solução dual viável que possibilite computar o custo reduzido, necessário para provar esta afirmação.
Um importante desenvolvimento proposto pelos autores é justamente um modelo linear capaz de encontrar eficientemente uma solução, mesmo que não seja ótima (o que, como argumentado por eles, é ainda mais eficiente).

Outro bom indicador do que pretendemos fazer é a semelhança de bons resultados recentes como \textcite{Hoeve2021Graphcoloringdecision} que utilizam ideias muito similares.

Técnicas interessantes também foram propostas por \textcite{Pessoa2021SolvingBinPacking} que apresentam um modelo genérico para resolver problema de roteamento.
Quando se adiciona um corte no \emph{branch-cut-and-price}, isso corresponde a variáveis no dual dificultando o subproblema de geração de colunas.
Os autores utilizam cortes de rank-\(1\) com memória limitada para melhorar tal processo.
Além disso, eles usam \textbf{propagação de etiquetas}, técnica comum na resolução de problemas de precificação que pode ser interessante para o nosso problema.

Tendo em vista que tais técnicas ainda não foram aplicadas ao problema de coloração, objetivamos principalmente aplicarmo-nas tais bem como desenvolver qualquer necessidade de modificação ou melhoria.
Além disso, consideramos estudar a adição de algoritmos de planos de corte, algo que já foi estudado brevemente na literatura, mas que ainda não viu grande impacto prático. \textcite{Schindl2005Somecombinatorialoptimization} introduziu alguns resultados poliédricos sobre a formulação de cobertura de conjuntos, \textcite{Hansen2009Setcoveringpacking} tentaram aplicar alguns cortes (sem muito sucesso) mas \textcite{Hulst2021branchpricecut} conseguiu avanços significativos em sua tese.

Por fim, como apresentado no começo desse projeto, diversos são os problemas similares ou generalizações bem como as aplicações de tais.
Assim, podemos também voltar nossas técnicas e implementações a tais variantes.
\end{document}