% Created 2023-01-10 ter 15:17
% Intended LaTeX compiler: pdflatex
\documentclass[11pt]{article}
\usepackage[utf8]{inputenc}
\usepackage[T1]{fontenc}
\usepackage{graphicx}
\usepackage{longtable}
\usepackage{wrapfig}
\usepackage{rotating}
\usepackage[normalem]{ulem}
\usepackage{amsmath}
\usepackage{amssymb}
\usepackage{capt-of}
\usepackage{hyperref}
\usepackage{todonotes}
\usepackage[portuges]{babel}
\usepackage{amsthm}
\usepackage{setspace}
\doublespacing
\newtheorem{prep}{Preposição}[section]
\author{Ieremies Vieira da Fonseca Romero}
\date{}
\title{Programação Linear Inteira e Combinatória Poliédrica do Problema de Coloração em Grafos}
\hypersetup{
 pdfauthor={Ieremies Vieira da Fonseca Romero},
 pdftitle={Programação Linear Inteira e Combinatória Poliédrica do Problema de Coloração em Grafos},
 pdfkeywords={},
 pdfsubject={},
 pdfcreator={Emacs 28.2 (Org mode 9.6)}, 
 pdflang={Portuges}}
\usepackage{biblatex}
\addbibresource{~/arq/bib.bib}
\begin{document}

\maketitle
\begin{abstract}
Histórico / importância do problema.
Usar PLI, em especial branch-and-price (geração de colunas) e branch-and-cut (planos de corte), no problema de coloração de grafos.
Objetivamos novas formulações, cortes e estratégias para melhor resolver as instâncias consolidadas da literatura.
\end{abstract}

\section{Introdução}
\label{sec:org3b390cb}
Dado um grafo não direcionado, o \textbf{problema de coloração de vértices} (do inglês, VCP) requer que designemos uma cor a cada vértice sem que este possua vizinhos de mesma cor e utilizemos o menor número possível de cores.
Sua NP-completude foi demonstrada por \textcite{Garey1979ComputersIntractabilityGuide}.

Como discutido por \autocite{Malaguti2010SurveyVertexColoring}, diversas são as aplicações do problema, como: agendamento \autocite{Leighton1979GraphColoringAlgorithm}, timetabling \autocite{Werra1985introductiontimetabling}, alocação de registradores \autocite{Chow1990prioritybasedcoloring}, comunicação de redes \autocite{Caprara2007PassengerRailwayOptimization} e alocação de banda \autocite{Gamst1986Somelowerbounds}.

As aplicações acima deixam claro que encontrar uma boa (ou até ótima solução) para o problema é crucial. Cenários reais muitas vezes lidam com centenas de milhares de vértices, tornando necessário também resolvermos em tempo hábil. \todo{isso ficou horrível}

\subsection{Modelo matemático}
\label{sec:orgf87b736}
\todo[inline]{conceitos que são citados mais para frente (eu vou voltar aqui para escrever)}
\subsection{Revisão bibliográfica}
\label{sec:org9896070}
\todo[inline]{eu não sei o que colocar aqui (ou se precisa existir essa parte) Todos os trabalhos que coloquei estão citado em outras partes do texto}
\subsection{Problemas similares}
\label{sec:org07a55b5}
\textcites{Cornaz2008oneonecorrespondence}[][]{Cornaz2017SolvingVertexColoring} usam técnicas de outros problemas para resolver o VCP.
\subsubsection{Bandwidth Coloring Problem}
\label{sec:org808ad98}
O problema de \textbf{Multicoloração de Banda} (BMCP) é a combinação de dois problemas: Coloração de Banda e Multicoloração.
No problema de \textbf{Coloração de Banda}, a diferença entre a cor de cada par de vértices adjacentes deve ser, ao menos, a distância entre os dois vértices. Assim, quando a distância é igual a \(1\), este problema é exatamente o VCP.
No problema de \textbf{Multicoloração} um valor \(w_i\) é designado a cada vértice e \(w_i\) cores devem ser alocadas a este, de forma que um par qualquer de vértices adjacentes não compartilhe nenhum cor em comum.

Assim, no BMCP, é necessário atribuir, para um vértice \(i\), \(w_i\) cores e para qualquer par de vértices adjacentes, cada combinação dois a dois de cores atribuídas a eles deve ter diferença maior que a distância entre os vértices.

Este problema permite que situações mais complexas que o VCP sejam modeladas, como o problema de alocação de frequência em telecomunicações \autocite{Aardal2007Modelssolutiontechniques}.

\subsubsection{Bounded VCP}
\label{sec:orgfa8dd4f}
Muitas vezes, o recurso que queremos alocar é limitado. Assim, podemos colocar um peso \(w_i\) em cada vértice e limitar a soma dos pesos dos vértices alocados a cada uma das cores, uma restrição de capacidade.
Este problema é conhecido como VCP Limitado ou Problema de Empacotamento com Conflito \textcite{Connolly1991KnapsackProblemsAlgorithms}.

Seja \(C\) a capacidade de cada cor, a restrição de capacidade é dada por
\[ \sum_{i=1}^n w_i x_{ih} \leq C, \quad h = 1, \dots, n\]
e pode ser adicionada à formulação de ASS para modelar o problema de BVCP.
\todo{Isso tá antes da formulação}
\subsubsection{WVCP}
\label{sec:org8f589e6}
\todo[inline]{to be done}
\subsubsection{Problemas similares}
\label{sec:org29e70e1}
\todo[inline]{esse foi um que esbarrei. Eu vou atrás de outros como b-coloring (, equitable) coloring (, rainbow) coloring?}
\begin{enumerate}
\item MSCP
\label{sec:org14b1d41}
O problema de Coloração de Soma Mínima nossa função objetivo é soma dos inteiros (cores) alocados a cada vértice.
\end{enumerate}
\section{Metodologia}
\label{sec:org7f9b27a}
\subsection{Programação Linear}
\label{sec:orgb1eee4c}
\textbf{Programação Linear} é uma técnica de otimização de problemas a partir da modelagem dos mesmos em \textbf{programas lineares}.
Nestes, definimos uma função objetivo, a qual queremos maximizar ou minimizar com suas variáveis sujeitas a um conjunto de restrições lineares (equações ou inequações lineares) \textcite{Chvatal1983LinearProgramming} . Um programa linear pode ser escrito da seguinte forma:
\begin{alignat*}{4}
& \omit\rlap{minize \quad \quad $\displaystyle cx$} \\
& \mbox{sujeito a}&& \quad & Ax & \geq b  & \quad &  \\
&                 &&       & x               & \in \mathbb{R}_+ &      &
\end{alignat*}

Para encontrar soluções viáveis com valores ótimos, conhecemos o algoritmo \textbf{simplex} que, apesar de ter complexidade exponencial, no caso médio executa em tempo polinomial.

\subsection{Programação linear inteira}
\label{sec:org9d5fb36}
Para alguns problemas, como o de coloração de grafos, não faz sentido falar em soluções fracionárias, afinal, não conseguimos designar ``meia cor'' a um vértice.
Para isso, restringimos as variáveis aos inteiros, fazendo assim um \textbf{Programa Linear Inteiro}. Caso apenas um subconjunto das variáveis possuam a restrição de integralidade, chamamos esse programa de linear misto.

O que a princípio pode parecer uma pequena alteração, torna o problema computacionalmente muito mais complexo.
Para encontramos boas soluções viáveis para esse tipo de programa, algoritmos como o simplex não são o suficiente.
Para isso, utilizamos técnicas como \textbf{branch-and-bound}, que consiste em dividir o problema em subproblemas menores e, durante o processo, encontrar limitantes que permitam diminuir o espaço de busca.

\subsection{Formulação clássica (atribuição)}
\label{sec:org9003dfb}
Sabemos que \(n\) cores são suficientes para colorir um grafo \(G\). Assim, podemos definir dois conjuntos de variáveis binárias: \(x_{ih}\) se o vértice \(i\) é colorido com a cor \(h\) e \(y_h\) se a cor \(h\) é utilizada. Dessa forma, definimos a seguinte formulação.
\begin{alignat*}{4}
\mathrm{(ASS)} \quad & \omit\rlap{minimize  $\displaystyle \sum_{i=1}^n y_h$} \\
& \mbox{sujeito a}&& \quad & \sum_{h=1}^n x_{ih}&= 1        & \quad & i \in V \\
&                 &&   & x_{ih} + x_{jh}    & \leq y_h &   & (i,j) \in E, h=1,\dots,n \\
&                 &&   & x_{ih}    & \in \{0,1\} &   & (i,j) \in E, h=1,\dots,n\\
&                 &&   & y_i       & \in \{0,1\} &   & i \in V
\end{alignat*}
Apesar de sua claridade e simplicidade, tal formulação vê pouca aplicação prática sem que apliquemos técnicas mais sofisticadas.

Esse fato se dá por dois motivos:
\begin{itemize}
\item Muitas soluções são simétricas umas às outras, já que as cores são indistinguíveis. Uma solução que utiliza \(k\) cores possui \(k\) permutações de cores do que é, efetivamente, a mesma solução.
\item A relaxação linear do modelo é extremamente fraca.
\end{itemize}

\textcites{MendezDiaz2006BranchCutAlgorithm}[][]{MendezDiaz2008CuttingPlaneAlgorithm} se dedicaram a resolver tais problemas.
\textcite{MendezDiaz2006BranchCutAlgorithm} adicionaram a restrição
\[ y_h \geq y_h+1 \quad h = 1, \dots, n-1 \]
que garante que a cor \(h+1\) só será utilizada se a cor \(h\) já estiver sendo.

Eles também acrescentaram diversas famílias de inequalidades válidas ao politopo do novo modelo que são adicionadas ao algoritmo de \emph{Branch-and-Cut} \todo{definir} para fortalecer a relaxação linear além de implementar a estratégia de branching proposta por \textcite{Brelaz1979Newmethodscolor} com resultados computacionais satisfatórios.
\todo[inline]{eu preciso mostrar as inqualidades?}

Já \textcite{MendezDiaz2008CuttingPlaneAlgorithm} apresentam mais duas variações da formulação ASS: uma onde a quantidade de vértices cuja cor \(h+1\) é atribuída não pode ser maior que a quantidade atribuída a cor \(h\) e outro onde conjuntos independentes são ordenados pelo menor índice e apenas a cor \(h\) pode ser atribuída ao \(h-\text{ésimo}\) conjunto.

\subsection{Formulação por representantes}
\label{sec:org00de8d5}
\autocite{Campelo2004CliquesHolesVertex} propuseram uma formulação baseada em representantes, na qual cada cor é representada por um vértice.
Para tal, utilizamos a variável binária \(x_{vu}\), para todo \(u, v \in V\) não adjacentes, a fim de representar se o vértice \(v\) é representante da cor de \(u\) e \(x_{vv}\) se \(v\) é o próprio representante de sua cor.
Seja \(\bar{N}(v)\) o conjunto de vértices não adjacentes de \(v\), esta formulação pode ser escrita como
\begin{alignat*}{4}
\mathrm{(REP)} \quad & \omit\rlap{minimize  $\displaystyle \sum_{v \in V} x{vv}$} \\
& \mbox{sujeito a}&& \quad & \sum_{u \in \bar{N}(v) \cup \{v\}} x_{uv}&= 1        & \quad & v \in V \\
&                 &&   & x_{vu} + x_{vw}    & \leq x_{vv} &   & v \in V, \forall e = (u,w) \in G[\bar{N}(v)] \\
&                 &&   & x_{vu}       & \in \{0,1\} &   & \text{ para todo par de vértices $u$, $v$ não adjacentes ou $v = u$}
\end{alignat*}
O primeiro conjunto de restrições garante que todo vétice terá extamente um representante.
O segundo garante que dois vértices adjacentes terão representantes diferentes.

Como \autocite{Campelo2008AsymmetricRepresentativesFormulation} discute, existem diversas soluções simétricas que apenas mudam o representante das cores sem alterar efetivamente a solução.
É proposto por eles acrescentar uma ordenação para que apenas o menor vértice podesse ser o representante.
Porém, este modelo possui um número exponencial de variáveis.
Os autores também apresentam diversas restrições válidas a fim de reforçar o modelo.

Por fim, \autocite{Campelo2008AsymmetricRepresentativesFormulation} se debruça sobre essa formulação, realizando a caracterização completa do politopo para algumas classes de grafos.
Experimentos computacionais foram feitos por \autocite{Jabrayilov2018NewIntegerLinear} mostrando a capacidade deste modelo de competir com as demais formulações.
\subsection{Formulação de cobertura de conjuntos (branch-and-price)}
\label{sec:orge1d7fd3}
Proposto por \textcite{Mehrotra1996ColumnGenerationApproach}, outra forma de entender o problema é imaginá-lo como um problema de cobertura de conjuntos onde os conjuntos disponíveis são os conjuntos independentes dos vértices.\todo{conjunto conjunto conjunto}

Assim, seja \(S\) a família de conjuntos impendentes do grafo \(G\), a variável binária \(x_s\) representa se o conjunto \(s \in S\) está sendo usado ou não na solução. Nossa formulação então se dá por:
\begin{alignat}{4}
& \omit\rlap{minimize  $\displaystyle \sum_{s \in S} x_s$} \nonumber \\
& \mbox{sujeito a}&& \quad & \sum_{s \in S: i \in s} x_{s}&\geq 1 & \quad & i \in V \label{rest9} \\
&                 &&   & y_s       & \in \{0,1\} &    & s \in S \nonumber
\end{alignat}
\todo[inline]{Explicação das restrições?}
Já essa formulação sofre de ter um número exponencial de variáveis, o que a torna impossível de implementá-la em ``SOLVERS'' convencionais como \emph{Gurobi}.

\textcite{Mehrotra1996ColumnGenerationApproach} propuseram um algoritmo de \emph{branch-and-price} \todo{definir} baseado na formulação de cobertura de conjuntos.
\todo[inline]{acho que aqui tem que ir a definição de branch-and-price}

O subproblema de geração de coluna caracteriza um \textbf{Problema de Conjunto Independente de peso máximo}.

\begin{alignat*}{4}
& \omit\rlap{maximize  $\displaystyle \sum_{i \in V} \pi_i z_i$} \\
& \mbox{sujeito a}&& \quad & z_i + z_j &\leq 1 & \quad & (i,j) \in E \\
&                 &&   & z_i       & \in \{0,1\} &    & i \in V
\end{alignat*}
onde \(z_i\) é uma variável binária que indica se o vértice \(i\) está incluso no conjunto independente e \(\pi_i\) é o valor ótimo da variável dual associado a restrição \ref{rest9}.
Tal problema pode ser resolvido de forma heurística para encontrar a coluna de custo reduzido com valor negativo.\todo{preciso explica o porquê disso?}
Em caso de soluções fracionárias, os autores sugerem uma estratégia \todo{explico qual?} que garante que os subproblemas continuam a ser de coloração de vértices e apenas requer que o grafo original seja alterado.

\begin{itemize}
\item \textcite{Malaguti2011ExactApproachVertex} propôs meta-heurísticas para inicialização e geração de colunas bem como novos esquemas de branching.
\item \textcite{Held2012Maximumweightstable} sugere técnicas para melhorar a estabilidade numérica
\end{itemize}

\textcite{Hansen2009Setcoveringpacking} propôs a formulação chamada de \textbf{Empacotamento de conjunto}.
\begin{alignat*}{4}
& \omit\rlap{minimize  $\displaystyle \sum_{s \in \Omega} (|s| - 1)x_s$} \\
& \mbox{sujeito a}&& \quad & \sum_{s \in \Omega: i \in s} x_{s}&\leq 1 & \quad & i \in V \\
&                 &&   & y_s       & \in \{0,1\} &    & s \in \Omega
\end{alignat*}
na qual \(\Omega\) é a família de conjuntos independentes com mais de um elemento.
Para essa formulação, seja \(z\) o valor da solução, a quantidade de cores usadas é igual \(k = n - z\).
Além disso, \textcite{Hansen2009Setcoveringpacking} demonstram a equivalência das formulações de SC e SP, bem como apresentam diversas famílias de inequalidades válidas que definem facetas\todo{definir}.

\begin{prep}
\textcite{Hansen2009Setcoveringpacking} Seja \(i \in V\), então a inequação correspondente /ref\{rest9\} define uma faceta se, e somente se, \(i\) não for dominado.
\end{prep}
\todo{definir dominado}

Os autores também apresentam resultados computacionais que não demonstram superioridade entre o trabalho deles em relação à \textcite{Mehrotra1996ColumnGenerationApproach}.
Por fim, duas técnicas de pré-processamento e um algoritmo de plano de corte \todo{definir}.
\subsection{Branch and bound usando DSATUR}
\label{sec:org88b284c}

\textcite{Brelaz1979Newmethodscolor} propôs o algoritmo guloso chamado de DSATUR, em que, a cada iteração, colorimos um vértice \(v\) como uma cor válida \todo{definir}.
Dizemos que o \textbf{grau de saturação} \todo{cromatico ou de saturação} de um vértice \(v\) numa coloração parcial \todo{definir} é a quantidade de cores distintas na sua vizinhança aberta \todo{definir}.
O DSATUR utiliza essa ideia para escolher, como próximo vértice a ser colorido, aquele com maior grau de saturação.

É possível utilizar essa ideia para melhorar nosso \emph{branch-and-bound}.
A cada ramificação, selecionamos o vértice com maior grau de saturação e criamos um problema para cada cor viável já utilizada, acrescentando uma ainda não utilizada.
\todo[inline]{talvez eu precise definir as notações de coloração parcial para isso ficar melhor}

Apesar disso, muitas vezes, diversos vértices possuem o mesmo grau de saturação, fazendo-se necessário implementar regras de desempate.
Dentre as propostas, temos:
\begin{itemize}
\item \textcite{Brelaz1979Newmethodscolor} utiliza o grau do vértice.
\item \textcite{Sewell1996improvedalgorithmexact} utiliza o vértice que maximiza o número de cores disponíveis para todos os vértices ainda não coloridos.
\item \textcite{Segundo2012newDSATURbased} incrementa na ideia anterior, mas apenas utilizando os vértices que estão sendo desempatados.
\end{itemize}
Em todos os casos acima, se mantiver algum empate, a ordenação lexigráfica é utilizada.

\textcite{Ternier2017ExactAlgorithmsVertex} implementa essas variações mostra que o proposto por \textcite{Sewell1996improvedalgorithmexact}, o qual se mostra o mais rápido, mesmo com maior complexidade computacional na regra de desempate, dado um bom limitante inferior inicial.

\textcite{Ternier2017ExactAlgorithmsVertex} apresenta novas variações para o algoritmo de \emph{branch-and-bound} usando DSATUR e novas regras de escolha de vértices com bons resultados em relação ao estado-da-arte.
\subsection{Ordenação parcial hibrida}
\label{sec:orgec90813}
Apresentado inicialmente por \autocite{Jabrayilov2018NewIntegerLinear} e posteriormente melhorado por \autocite{Jabrayilov2022StrengthenedPartialOrdering}, utilizamos um misto de ordenação parcial da união entre os vértices e as cores disponíveis e o modelo de atribuição.
Dizemos que o vértice \(v\) é colorido com a cor \(h\) se \(h-1 \succ v\) e \(h \nsucc v\) (no caso de \(h=1\), se \(h \nsucc v\)). \todo{definir H como upper bound}
Além disso, nesse modelo, é escolhido um vértice arbitrário \(q\).
A formulação segue:


\begin{alignat*}
\mathrm { (POPH) } \quad & \omit\rlap{minimize  $\displaystyle 1+\sum_{1 \leq h \leq H} g_{h, q}$} \\
& \mbox { sujeito a } && \quad & g_{H, v} &=0          & \quad & \forall v \in V \\
&                     &&   & x_{v, 1} &=1-g_{1, v} & \quad & \forall v \in V \\
&                     &&   & x_{v, h} &=g_{h-1, v}-g_{h, v} & \quad & \forall v \in V, h=2, \ldots, H \\
&                     &&   & x_{u, 1}+x_{v, 1} &\leq g_{1, q} & \quad & \forall u v \in E \\
&                     &&   & x_{u, h}+x_{v, h} &\leq g_{h-1, q} & \quad & \forall u v \in E, h=2, \ldots, H \\
&                     &&   & g_{h, q}-g_{h, v} &\geq 0 & \quad & \forall v \in V, h=1, \ldots, H \\
&                     &&   & g_{h+1, q}-g_{h, v} &\geq 0 & \quad & \forall v \in N(q), h=1, \ldots, H-1 \\
&                     &&   & x_{v, h}, g_{h, v} &\in\{0,1\} & \quad & \forall v \in V, h=1, \ldots, H \text {. } \\
&                     &&   &
\end{alignat*}
\todo[inline]{vou reformatar essa explicação}
O primeiro conjunto de restrições garante que nenhum vértice é maior que a cor H.
O segundo e terceiro correlacionam as variáveis de ordenação parcial com as de atribuição.
O quarto e quinto garantem que dois vértices adjacentes não são coloridos com a mesma cor.
Já a sexta, força que \(q\) seja o vértice com a maior cor.
A sétima é utilizada para reforçar.

Segundo os resultados experimentais de \autocite{Jabrayilov2022StrengthenedPartialOrdering}, essa formulação domina os modelos anteriores nas instâncias DIMACS \autocite{GraphColoringInstances} esparsas (densidade \(\frac{2|E|}{|V|(|V|-1)} \leq 0.1\)).

\subsection{Diagrama de decisões binárias ordenadas}
\label{sec:org4f68ac9}
\autocite{Hoeve2021Graphcoloringdecision}
\subsection{Estado da arte}
\label{sec:org005a0e9}
\textcite{Jabrayilov2018NewIntegerLinear} implementam as abordagens acima e mostra não haver uma dominância clara entre nenhuma delas.
Apesar disso, nos seus testes, ordenação parcial se sai melhor em grafos esparsos enquanto a formulação de representantes se sai melhor em grafos densos.
\section{Objetivos}
\label{sec:org8120bf4}
Neste projeto, objetivamos propor novos modelos de PLI para dominação romana e suas variantes explorando técnicas como \emph{branch-and-cut} e \emph{branch-and-price}.
Além disso, estudaremos a possibilidade de novos cortes e limitantes para as formulações.

\todo[inline]{aqui a minha ideia é apresentar esse tal de ferramental moderno e as ideias mais recentes que podemos aplicar}
\textcite{Lima2022Exactsolutionnetwork}
\textcite{Pessoa2021SolvingBinPacking}
\section{Cronograma}
\label{sec:org6b6bbbf}
BEPE indicar umas possibilidades de nomes. Manuel Iori.

\section{Material e método}
\label{sec:org1a7079f}
Para o desenvolvimento do projeto, o aluno utilizará-se de artigos e materiais de consulta disponibilizados pela UNICAMP de maneira gratuita, grande parte desses de forma online ou por meio da Biblioteca do Instituto de Matemática, Estatística e Computação Científica.

Ademais, serão realizados encontros semanais entre o aluno e o orientador para debater os conteúdos estudados e acompanhar o progresso do projeto.

\section{Avaliação dos resultados}
\label{sec:org84800e3}
Os algoritmos e modelos propostos serão comparados com as instâncias presentes na literatura, como as \textcite{GraphColoringInstances} e, caso necessário, novas instâncias poderão ser geradas.

Os resultados dos experimentos computacionais serão comparados utilizando técnicas como \textbf{Performance Profile} demonstrado por \textcite{Dolan2002Benchmarkingoptimizationsoftware}.

RELATÓRIOS

\printbibliography
\end{document}